\documentclass[man,biblatex,apacite]{apa6}
\usepackage{array}
\usepackage{float}
\usepackage{graphicx}
\usepackage[american]{babel}
\usepackage{amsmath}
\usepackage{comment}
\usepackage{multirow}
\usepackage{breqn}
\usepackage{rotating}
\usepackage{pdflscape}
\usepackage{listings}
\usepackage{caption}
\usepackage{subcaption}
\usepackage{alltt}




\author{Yi Chen}
\affiliation{Teachers College, Columbia University}
\title{Research Methods In Social Psychology: Assignment 1}
\shorttitle{Assignment 1}
\abstract{In this assignment, I will gererate a research idea about human behavior (self-directed e-learning) and briefly discuss the major research components (e.g., hyphothesis, independent and dependent variables, modeiators and mediators).
}

\keywords{human behavior, general idea, research components, self-directed learning}

\begin{document}
\maketitle
\section{Introduction}
Acknowledging the relation between individual differences and education has a very long history \cite{Shute2003}. 
Back to 1970s, Dick Snow discussed this issues in his study on how individual aptitudes played out in a different educational setting. 
As he criticized that "the designers of policy and practice often ignore the lessons of differential psychology by trying to impose an one-size-fits-all solution even though individuals are different." 
Thanks to the development of network technology and electronic communications, the digital revolution in education make autonomous, customized, and adaptive learning possible in reality. 
In the light of the digital revolution, self-directed learning becomes increasingly important in the Information Age.

Self-directed learning (SDL) is learning in which the conceptualization, design, conduct, and evaluation of a learning project are directed by the learner \cite{Brookfield2006}.
Other terminologies have been used in this area of research: independent learning \cite{Brookfield2006}, self-initiated learning \cite{Levinsen2011}, and self-motivated learning \cite{Bonarini2006}.

In this assignment, the main research topic is about \textbf{self-directed learning}.
In particular, \textbf{self-directed e-learning}.
More concretely, I want to know: \textbf{what is the factors that influence the performance of self-directed e-learning, and how does these factors impact?}
The reason that I am interested in self-directed learning can be summarized as the following three points:
\begin{enumerate}
    \item SDL may well be the most prominent and well-researched topic in the field of adult education  \cite{Brookfield1993,Brockett2019,Garrison1997}.
          According to \citeA{Tough1971}, 90\% of all adults were claimed to conduct at least one SDL project each year, spending an average of 100 hours on this effort.
          Now, the importance of SDL is increasing because of the popularity of personalized digital applications in education (e.g., Massive online open courses).
    \item As a data science researcher in EdLab of Teachers College, I accumulated some experience of learners' behavior on multiple digital learning platforms.
          How to analysis, test, and measure the self-directed learning with the strong support from the theory in education and psychology is the question that I alway have.
    \item In methodology, the self-reported nature of SDL research attracts many critiques \cite{Brookfield2006}. 
          \citeA{Stockdale2001}, and \citeA{Brookfield2006} believed SDL study needs revitalization in using more efficient and accurate technologies to synthesize and measure the findings.
          The technologies in digital learning context indeed provides the opportunities in research.
\end{enumerate}

\section{Research Components}
\subsection{Independent Variables, and Dependent Variables}
The dependent variable in this research will be the performance of self-directed e-learning.
The measurement of performance can be multidimensional (e.g., engagement level, skill development, or values clarification).
These measurements can also be summative or formative assessments.
For example, we can measure the engagement level through the visiting time, number of click, and number of comments on the platforms.

The potential independent variables are the factors which may impact the performance of self-directed e-learning.
Based on the pervious research, the influence factors can be: peer-network \cite{Brookfield1985}, human resources \cite{Candy1991,Brockett2019}, and learning communities \cite{Abdullah2001}.
These factors are widely discussed in the traditional learning environment (e.g., classroom).
However, they are not common considered in the digital environment.
Thus, it will not be unreasonable that some of the factors may not critical anymore. 
Or, there are some other factors exist.


In this research, the independent variable that I want to focus on is: learning resources.
I will take the usage of learning resource as the mediator ("useful" or "not useful"). 
The independent variable is the the influence from professor and influence from the author.
For each learning resource, we classify whether this resource is recommended by professor or not, and whether this is written by the famous researchers or not.
These two factors influence the usefulness of the learning resource in students' perspective.
These two factors are measured in general, not for a specific item.

\subsection{Moderator and Mediators}
The moderator in this research will be the learning environment: physical or digital.
The representatives learning environment for physical self-directed learning is library.
One of the representatives digital learning environments could be Khan Academy.
So the same person could take the same course (e.g., machine learning) online or offline.

In terms of mediators, I think it is the usage of learning resource.
This can be whether the students feel the learning resource is useful or not.


\subsection{Hypothesis}
The basic hypothesis of this research can be summarize as following:
In physical learning environment, I think the influence of professor and author all have a positive effects on the performance of learning.
This indicates that if the students follow the recommendation from professors and the famous researchers, they are more likely to be successful.
The same pattern is expected to appear for the digital environment.
However, we assume that the positive effects of the influence of professor will not have big differences under different learning environments.
While, the influence from the author will have more significant effects in digital environment.

\section{Note}

Part of the study in this assignment is based on my pervious research named "Challenges and Opportunities of Using Recommendation System in Self-directed Learning Education".
This research has been submitted to American Educational Research Association (AERA, 2020),  and is currently  under review by my supervisor in EdLab.

\clearpage
\bibliography{library}
\end{document}
