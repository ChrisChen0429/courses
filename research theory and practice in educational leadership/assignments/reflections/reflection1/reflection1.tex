\documentclass[doc,biblatex,apacite]{apa6}
\usepackage{array}
\usepackage{float}
\usepackage{graphicx}
\usepackage[american]{babel}
\usepackage{amsmath}
\usepackage{comment}
\usepackage{multirow}
\usepackage{breqn}
\usepackage{rotating}
\usepackage{pdflscape}
\usepackage{listings}
\usepackage{caption}
\usepackage{subcaption}
\usepackage{alltt}




\graphicspath{ {./image/} }
\author{Yi Chen}
% \shorttitle{ }
\affiliation{Teachers College, Columbia University}
\title{Leadership-Management Relationship: A Education Perspective}
\abstract{
 The goal of this reflection paper is to clarify the relationship between educational leadership and educational management.
 I will provide a summary of the second-week readings, and points out the open questions that I want to discuss under this topic.
}

\keywords{Education, Leadership, Management, Relationship}

\begin{document}
\maketitle

\section{Comprehension}

One of the most critical topics in second-week readings is about the relationship between leadership and management. 
From the ancients (the Greeks) to the information era, the definition of leadership is not the same. 
Socrates believes there is no natural difference between the leaders for private affairs and public concerns \cite{XenophontransJ.S.Watson1875}.
Although Antisthenes has no experience in military service, he is still "able to find out the best masters," knows to "punish the bad and honor the good," and "zealously seek everything that may conduce to victory."
Socrates focus on the competencies, characteristics, and vision, which he believes is transferable or (to some extent) context-independent.
There is no clear distinction of leadership and management from from the his opinion.

However, \citeA{Kotter2013} is unsatisfied with the confusion of "management" and "leadership."
Kotter's idea, under the definition of \citeA{Simonet2013}, is the bipolarity or bi-dimensionality relationship.
Many other researchers also hold the similar concept.
For example, \citeA{Bennis1992} said "leaders are those who do the right thing, while managers are those who do the right thing."
Besides, \citeA{Firestone1996} claimed that "the task of management is to maintain order, while leadership is to promote change."
Apart from that, \citeA{Simonet2013} pointed out that the relationship between leadership and management can also be: uni-dimensionality (e.g., Socrates) and hierarchical (both management contains leadership, or leadership contains management).

\section{Evaluation}
However, the readings did not discuss the relationship between management and leadership in the context of education.
The definition we have for these two terms ("what") will directly determine the way we manage and lead in practice ("How").
\begin{itemize}
    \item Educational leadership once belonged to one research filed of educational management \cite{Richmon2003}. Some traditional educational researchers still believe education leadership is a branch of educational management.
    \item Educational leadership and educational management are the same subjects. Leadership includes administration and management, while managers tend to conceive leadership by definition \cite{Hodgkinson1991}.
    \item Educational leadership includes educational management. 
          Educational leaders are regarded as leaders in the teaching process, leading professional development and improving school quality to meet the internal and external needs of schools \cite{Levin2006}.
    \item Educational leadership is a cross-cutting but not entirely identical field with educational management. 
    In 2003, the British journal Educational Management Administration even changed its name into Educational Management Administration \& Leadership.
\end{itemize}

Educational leadership is highly influenced by educational management, along with the influence of anthropology, psychology, sociology, and many other disciplines.
The relationship between educational leadership and educational management may also change over time.
In this information era, research and practice of educational leadership has to face the revolution in the education.
However, these questions have not been covered in the readings.
\begin{itemize}
    \item In the first two chapters of \citeA{Firestone2005}, they mentioned some research methodology (e.g., experiment and quasi-experiment). 
          What is the potential of Big data technology in educational leadership research?
    \item \citeA{Simonet2013} analysis educational leadership focusing on the critical competencies. 
          However, the improvement of the policy system, processing framework, and working environments are not emphasized.
          What is the role of these superstructural factors in the practice of educational leadership?
          And, how we can design the new superstructure in education for the next generation of learners?
\end{itemize}

\section{Application}
Almost all the readings for this week focus on personal competencies, characteristics, and vision.
It is not doubted that leadership or management is an art of behavior which needs learning and practicing.
For example, Ms. Johnson is a good leader based on her ability and experience.
However, the failure of Turnaround school projects, from my perspective, rooted in the non-interactive working process and formalistic working environments.
The superstructural factors (policy system, processing framework, and working environments) may play a more significant effect.

When I was reviewing the last week reading about the Turnaround Schools project \cite{Peurach2010}, I immediately thought of Scrum.
Scrum is a lightweight, iterative, and incremental framework for effective team collaboration on intricate knowledge work \cite{Schwaber2004}. 
It is a simple, agile process, which emphasis self-organizing, cross-function, and continual improvement.
As a data science researcher in an education lab, I use Scrum every day for all the product or research projects to collaborate with other lab members.
I believe proper management comes from good leadership, while good leadership relies on proper management.

The participants of Scrum (broadly) includes product owner (figure out "what" to do to maximize the value of the product), 
scrum master (figure out "how" to do to promote and support the use of Scrum), 
moreover, the development team (self-organizing in pursuit of delivering value; sometimes can be divided into build team and test team).
According to the idea of \citeA{Kotter2013}, product owner here is more like the leader, while scrum master is more like a manager.
Through the framework of Scrum, manager, and leader work together to fulfill the needs of stockholders.
Additionally, the product owner and scrum master usually involved in the development team.
Instead of the command-and-control waterfall, it makes every project as a collaborative learning opportunity.
\begin{figure}[hbt!]
    \centering
    \includegraphics[width=10cm]{Scrum.png}
    \caption{Scrum}
  \end{figure}

Even though Scrum is an initial emphasis on software development, it has great potential in many scenarios beyond IT \cite{RebeccaSullivan}.
For example, J (a physics teacher)  uses Scrum  (software development framework) to keep his students accountable and on-task during group projects. 
Every morning, students meet and report to their group what they have accomplished, what they will work on, and current worries (this step is called daily standup). 
The poster is used to visualize their progress: pending, in-process, block, and finished.
J is convinced that in order for real student-centered learning to occur, the teacher must occupy a managerial role.




\section{Conclusion}
The current generation of learners utilizes electronic technologies and has more choices than ever before when it comes to accessing digital resources inside or outside of traditional learning environments. 
For example, distance education, digital electronic learning, Massive Online Open Courses (MOOCs),  and many others deliver courses, programs, or degrees entirely online.
The digital revolution in education makes leadership even much more challenging. However, it also brings opportunities.
No matter management or leadership is the same or different.
The leader in the future has to make use of the techniques (like Scrum) and work with other managers in their daily practices.
Additionally, researchers should also pay attention to the massive data that we generate and collect for better analysis, measurement, and evaluation.

\bibliography{library}


\end{document}
